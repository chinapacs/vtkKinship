%
% Complete documentation on the extended LaTeX markup used for Insight
% documentation is available in ``Documenting Insight'', which is part
% of the standard documentation for Insight.  It may be found online
% at:
%
%     http://www.itk.org/

\documentclass{InsightArticle}

\usepackage[dvips]{graphicx}

%%%%%%%%%%%%%%%%%%%%%%%%%%%%%%%%%%%%%%%%%%%%%%%%%%%%%%%%%%%%%%%%%%
%
%  hyperref should be the last package to be loaded.
%
%%%%%%%%%%%%%%%%%%%%%%%%%%%%%%%%%%%%%%%%%%%%%%%%%%%%%%%%%%%%%%%%%%
\usepackage[dvips,
bookmarks,
bookmarksopen,
backref,
colorlinks,linkcolor={blue},citecolor={blue},urlcolor={blue},
]{hyperref}


%  This is a template for Papers to the Insight Journal. 
%  It is comparable to a technical report format.

% The title should be descriptive enough for people to be able to find
% the relevant document. 
\title{Explicit Deformable Model in VTK}

% 
% NOTE: This is the last number of the "handle" URL that 
% The Insight Journal assigns to your paper as part of the
% submission process. Please replace the number "1338" with
% the actual handle number that you get assigned.
%
\newcommand{\IJhandlerIDnumber}{Xxxx}

% Increment the release number whenever significant changes are made.
% The author and/or editor can define 'significant' however they like.
\release{1.00}

% At minimum, give your name and an email address.  You can include a
% snail-mail address if you like.
\author{J\'er\^ome Velut$^{1}$}
\authoraddress{$^{1}$France, jerome.velut@gmail.com}

\begin{document}

%
% Add hyperlink to the web location and license of the paper.
% The argument of this command is the handler identifier given
% by the Insight Journal to this paper.
% 
\IJhandlefooter{\IJhandlerIDnumber}


\ifpdf
\else
   %
   % Commands for including Graphics when using latex
   % 
   \DeclareGraphicsExtensions{.eps,.jpg,.gif,.tiff,.bmp,.png}
   \DeclareGraphicsRule{.jpg}{eps}{.jpg.bb}{`convert #1 eps:-}
   \DeclareGraphicsRule{.gif}{eps}{.gif.bb}{`convert #1 eps:-}
   \DeclareGraphicsRule{.tiff}{eps}{.tiff.bb}{`convert #1 eps:-}
   \DeclareGraphicsRule{.bmp}{eps}{.bmp.bb}{`convert #1 eps:-}
   \DeclareGraphicsRule{.png}{eps}{.png.bb}{`convert #1 eps:-}
\fi


\maketitle


\ifhtml
\chapter*{Front Matter\label{front}}
\fi


% The abstract should be a paragraph or two long, and describe the
% scope of the document.
\begin{abstract}
\noindent
This document describes a set of classes\footnote{It is a subset of the
vtkKinship library \url{http://github.com/jeromevelut/vtkKinship}} that 
design a generic explicit deformable model in VTK. The iterative mechanism
is first introduced through an inheritance of the vtkPolyDataAlgorithm class. 
This vtkIterativePolyDataAlgorithm is then a based for an implementation of the
deformation. 
\end{abstract}

\IJhandlenote{\IJhandlerIDnumber}

\tableofcontents

The deformable models are part of a wide family of computer graphics methods.
They were initially proposed in the context of real physics simulation, 
in particular for solid and soft body deformations \cite{TERXX}. One of the
applications concerns the segmentation task in image processing:
The snakes \cite{KAS87} are often cited as the seminal work that induced the
huge amount of existing paper today \cite{MON01}. The main idea was to minimize
an energy computed under a curve evolving in an image domain. The proposed 
implementation was a discretization of the steepest descent algorithm, which
led to an iterative displacement of each point related to a trade-off between
internal and external forces. The explicit deformable models refer to this 
iterative deformation of the geometry, as opposed to implicit deformable
models \cite{}.

In this paper, we focus on the iterative design of an explicit deformable 
model. First, we present the vtkIterativePolyDataAlgorithm class, that 
inherits from vtkPolyDataAlgorithm. It allows a VTK filter to iteratively
process the input until a number of iteration is reached. A step-by-step
mechanism is also implemented, bringing useful interactive capabilities
-especially inside ParaView-. Second, a deformable model is designed through
a class derived from vtkIterativePolyDataAlgorithm. It has been kept as simple
as possible, e.g the user is responsible of the forces computation and the
initialisation. Finally, a ParaView plugin is provided and a short example 
is given in which the animation feature is used in order to interact with the 
deformable model.
%
\section{Making PolyData algorithms iterative}
\subsection{basics}
The design of the VTK library is based on a pipeline of data processing. The
modification of an input anywhere in the pipeline triggers the execution of the
whole remaining, downward processes. An obvious consequence is the 
impossibility to plug the output of a filter directly to an upward input 
'as is': this feedback connection will create in many cases an infinite loop.
However, it is possible to do almost everything in the execution method, unless
the inputs are not modified.
%
\subsection{Caching the input}
%
\section{Explicit deformable model}

\section{Paraview plugin}


\section{Software Requirements}

You need to have the following software installed:

% The {itemize} environment uses a bullet for each \item.  If you want the 
% \item's numbered, use the {enumerate} environment instead.
\begin{itemize}
  \item  VTK-5.4
  \item  CMake-2.8
  \item  ParaView-3.8 (optional)
\end{itemize}

\appendix

%%%%%%%%%%%%%%%%%%%%%%%%%%%%%%%%%%%%%%%%%
%
%  Insert the bibliography using BibTeX
%
%%%%%%%%%%%%%%%%%%%%%%%%%%%%%%%%%%%%%%%%%

\bibliographystyle{plain}
\bibliography{references}


\end{document}

